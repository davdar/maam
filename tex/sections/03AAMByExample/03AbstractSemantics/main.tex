Now that we have designed an abstract state space--albeit highly
parameterized--we must turn to implementing its semantics.
%
The goal in this entire exercise is to develop the abstract state space and
semantics in such a way that concrete and abstract interpreters can be derived
from a single implementation.

%--%

Following the structure of the concrete semantics, we adapt each function to
the new abstract state space.
%
\h|op| is now entirely implemented by the designer of the analysis as part of
the \h|Delta| interface.
%
\h|atom| must follow another level of indirection in variable lookups, and must
account for nondeterminism.
%
\haskell{sections/03AAMByExample/03AbstractSemantics/00Atom.hs}
%
\h|call| must use the eliminators provided by \h|Delta d| to guide control
flow.
%
\haskell{sections/03AAMByExample/03AbstractSemantics/01Call.hs}
%
\h|bindMany| must join stores when binding to addresses, in case the address is
already in use.
%
\haskell{sections/03AAMByExample/03AbstractSemantics/02BindMany.hs}
%
Finally, \h|step| must advance time as it advances each \h|Call| state, and
\h|exec| is modified to a collecting semantics.
%
\haskell{sections/03AAMByExample/03AbstractSemantics/03Step.hs}
%
The collecting semantics keeps track of all previously seen states.
%
% TODO: reference kleene fixpoint theorem.  also motivate this idea
%       better/earlier
%
Keeping track of previous states turns exec into a monotonic function, which
means its iteration will terminate if the instantiated state space is finite.
