\label{section:correctness}

Novel in this work is the justification that a monadic approach to abstract
interpretation can produce sound and complete abstract interpreters by
construction.
%
As is traditional in abstract interpretation, we capture the soundness and
correctness of derived abstract interpreters by establishing a Galois
connections between concrete and abstract state spaces, and by proving an order
relationship between concrete and abstract transition functions.

We remind the reader of the definition of Galois connection:
\begin{lstlisting}
Galois (A B : Type) :=
  $\alpha$ : A $\nearrow$ B
  $\gamma$ : B $\nearrow$ B
  inverses : $\alpha \circ \gamma \sqsubseteq$ id $\sqsubseteq \gamma \circ \alpha$
\end{lstlisting}
where $A \nearrow B$ is written to mean the \textit{monotonic} function space
between A and B, namely [f : A $\nearrow$ B] means [$\exists$ f' : A $\to$ B
$\and$ $\forall$ x y, x $\sqsubseteq$ y $\to$ f x $\sqsubseteq$ f y].

% We further generalize the notion of Galois connection to the Kleisli category:
% \begin{lstlisting}
% KleisliGalois m A B | Monad m :=
%   $\alpha$ : A $\nearrow$ m B
%   $\gamma$ : B $\nearrow$ m A
%   inverses : $\alpha \circ \gamma \sqsubseteq$ id $\sqsubseteq \gamma \circ \alpha$
% \end{lstlisting}
% where id and $\circ$ are interpreted in the Kleisli category.
% 
% [Note why we need the Kleisli generalization.  It's related to wanting to
% distinguish between computable and non-computable functions, and finite and
% infinite sets.]
% 
% Lemma: KleisliGalois m A B $\to$ Galois (A $\nearrow$ m A) (B $\nearrow$ m B).
% %
% (This is the analogue of [Galois A B $\to$ Galois (A $\nearrow$ A) (B $nearrow$
% B)] in the traditional framework.)

Next we introducing the concept of a \textit{functorial Galois connection},
which establishes a Galois connection between functors:
\begin{lstlisting}
FunctorialGalois ($m_1$ $m_2$: Type -> Type) :=
  mapGalois : Galois A B -> Galois ($m_1$ A) ($m_2$ B)
GaloisTransformer (t : (Type -> Type) -> Type -> Type) :=
  liftGalois : FunctorialGalois $m_1$ $m_2$ -> FunctorialGalois (t $m_1$) (t $m_2$)
\end{lstlisting}
and we say (m : Type $\to$ Type) is a GaloisFunctor if it is proper in the
FunctorialGalois relation:
\begin{lstlisting}
GaloisFunctor m := FunctorialGalois m m
\end{lstlisting}

We prove the following lemmas:
\begin{lstlisting}
GaloisTransformer (StateT S)
GaloisTransformer SetT
GaloisFunctor Set
\end{lstlisting}

For a language $L$ and two monads $m_1$ and $m_2$, the concrete and abstract
state spaces in such a setting are $ss_{m_1} L$ and $ss_{m_2} L$, and the step
functions are (transition $step_{m_1}$) and (transition $step_{m_2}$). To
establish the relation (transition $step_{m_1}$) $\sqsubseteq$ (transition
$step_{m_2})$ we first enrich the (MonadStateSpace m) predicate to establish
an isomorphism between m and $ss_m$:
\begin{lstlisting}
MonadStateSpace m :=
  ss : Type -> Type
  transition : $\forall$ A B, Iso (A $\nearrow$ m B) (ss A $\nearrow$ ss B)
\end{lstlisting}
where
\begin{lstlisting}
Iso A B :=
  to : A $\to$ B
  from : B $\to$ A
  inverses : to $\circ$ from = id = from $\circ$ to
\end{lstlisting}

Given this stronger connection between monadic actions and their induced
state space transitions, we can now prove:
\begin{lstlisting}
FunctorialGalois $m_1$ $m_2$
MonadStateSpace $m_1$
MonadStateSpace $m_2$
step : $\forall$ m, L $\nearrow$ m L
----------------------------
Galois ($ss_{m_1}$ L) ($ss_{m_2}$ L)
$transition_{m_1}$ step : $ss_{m_1}$ L $\to$ $ss_{m_1}$ L
$transition_{m_2}$ step : $ss_{m_2}$ L $\to$ $ss_{m_2}$ L
$transition_{m_1}$ step $\sqsubseteq$ $transition_{m_2}$ step
\end{lstlisting}
which establishs the Galois connection between both the concrete and abstract
state spaces and interpreters interpreters induced by $m_1$ and $m_2$.
%
[... fancy diagram that sketches the proof ...]

Going back to our example, the only thing left to prove is that our step
function is monotonic.
%
Because \textbf{step} is completely generic to an underlying monad, we must
enrich \textbf{all} interfaces that we introduced in Section ? to carry
monotonic functions.
%
Generalizing these interfaces, proving that the suppliers of the interfaces
(the monad transformers) meet the monotonicity requirements, and proving that
the client of the interfaces (the step function) are all monotonic is a
systematic exercise in applying the functionality of the \_$\sqsubseteq$\_
relation.
%
We automate the proofs of such theorems using the Coq proof assistant (see
Appendix).
%
However we note that such a property can be shown to hold by construction for
all morphisms in an embedded logic (as can be done for equality) and therefore
it should come as no surprise that if all arrows are changed from $\to$ to
$\nearrow$ then everything ``just works out''.

Once it is established that our step function is monotonic, the proof burden
remains to establish [inc-concrete $\sqsubseteq$ inc-abstract]. [...]

We give a proof that heap widening is sound and less precise than heap cloning
by proving [FunctorialGalois (StateT S Set) (SetT (State S))]. [...]

Now we have an end-to-end proof that:
\begin{lstlisting}
$\text{transition}\; \text{step}_{\langle\text{m}_1, \text{inc-concrete}\rangle}$
$\sqsubseteq$
$\text{transition}\; \text{step}_{\langle\text{m}_1, \text{inc-abstract}\rangle}$
$\sqsubseteq$
$\text{transition}\; \text{step}_{\langle\text{m}_2, \text{inc-abstract}\rangle}$
\end{lstlisting}
or
\begin{lstlisting}
$\langle$concrete semantics$\rangle$
$\sqsubseteq$
$\langle$abstract semantics$\rangle$
$\sqsubseteq$
$\langle$abstract semantics w/heap widening$\rangle$
\end{lstlisting}
%
Proving this uses the facts [FunctorialGalois $m_1$ $m_2$, MonadStateSpace
$m_1$, MonadStateSpace $m_2$]--facts we get for free from the framework through
composition of monad transformers--and [inc-concrete $\sqsubseteq$
inc-abstract]--a proof obligation of the user of the framework.

%%%%%%%%%%%%%%%%%%%% Expressivity and Modularity %%%%%%%%%%%%%%%%%%%%


